\documentclass{article}
% \usepackage{corl_2022} % Use this for the initial submission.
\usepackage[final]{corl_2022} % Uncomment for the camera-ready ``final'' version.
\usepackage{times} % assumes new font selection scheme installed
\usepackage{brian}
\graphicspath{{../images/}}
% \usepackage{biblatex}

% Bibliography
% \usepackage[style=ieee,doi=false,isbn=false,url=false,eprint=false]{biblatex}
% \AtEveryBibitem{\clearlist{language}}  % remove language field from bib entries
% \addbibresource{references.bib}
% \renewcommand*{\bibfont}{\small}


\title{Extended Dynamic Mode Decomposition with Jacobian Residual Penalization
for Learning Bilinear, Control-affine Koopman Models}

\author{
  Brian E. Jackson \\
  Robotics Institute \\
  Carnegie Mellon University\\
  \texttt{brianjackson@cmu.edu} \\
  \and
  Jeong Hun Lee \\
  Robotics Institute\\
  Carnegie Mellon University\\
  \texttt{jeonghunlee@cmu.edu} \\
  \and
  Kevin Tracy \\
  Robotics Institute\\
  Carnegie Mellon University\\
  \texttt{ktracy@cmu.edu} \\
  \and
  Zachary Manchester \\
  Robotics Institute\\
  Carnegie Mellon University\\
  \texttt{zacm@cmu.edu} \\
}

\begin{document}
\maketitle

\section{Introduction}

    Controlling complex, underactuated, and highly nonlinear autonomous systems remains an
    active area of research in robotics, despite decades of previous work exploring
    effective algorithms and the development of substantial theoretical analysis. Classical
    approaches typically rely on local linear approximations of the nonlinear system, which
    are then used in any of a multitude of linear control techniques, such as PID, pole
    placement, Bode analysis, H-infinity, LQR, or linear MPC.  These approaches only work
    well if the states of the system always remain close to the linearization point or
    reference trajectory. The region for which these linearizations remain valid can be
    extremely small for highly nonlinear systems.  Alternatively, model-based methods for
    nonlinear control based on optimal control have shown great success, as long as the
    model is well known and an accurate estimate of the global state can be provided. These
    model-based techniques leverage centuries of insight into dynamical systems and have
    demonstrated incredible performance on extremely complicated autonomous systems 
    \todo{add citations for some recent work}.  On the other hand, data-driven techniques
    such as reinforcement learning have received tremendous attention over the last decade
    and have begun to demonstrate impressive performance and robustness for complicated
    robotic systems in unstructured environments \todo{cite a few RL results, including
    Marco Hutter's recent stuff}. While these approaches are attractive since they require
    little to no previous knowledge about the system, they often require large amounts of
    data and fail to generalize outside of the domain or task on which they were
    ``trained.''
    
    In this work we propose a novel method that combines the benefits of model-based and
    data-driven methods, based on recent work  applying Koopman Operator Theory to
    controlled dynamical systems \todo{cite a few Koopman papers,  including the CalTech
    ones}.  By leveraging data collected from an unknown dynamical system along with
    derivative information from an approximate analytical model, we can efficiently learn a
    bilinear representation of the system dynamics that performs well when used in
    traditional model-based control techniques such as linear MPC. The result is nonlinear
    dynamical system that is expressive enough to capture the full nonlinear dynamics across
    a broad range of the state space, but has a well-defined structure that is very amenable
    to specialized optimization-based controllers. By leveraging information from an
    analytical model, we can dramatically reduce the number of samples required to learn a
    good approximation of the true nonlinear dynamics. 
    
    \todo{bullet out contributions.}
    
    \todo{add summary of paper layout.}

\section{Background and Related Work} \label{sec:Preliminaries/Background}

  \subsection{The Koopman Operator}
  The theoretical underpinnings of the Koopman operator and its application to dynamical
  systems has been extensively studied, especially within the last decade \todo{cite some
  important theoretical works on Koopman theory here}. Rather than describe the theory in
  detail, we highlight the key concepts employed by the current work, and defer the
  motivated reader to the existing literature on Koopman theory.

  We start by assuming we have some discrete approximation a of controlled nonlinear,
  time-dynamical system whose underlying continuous dynamics are Lipschitz continuous:
  \begin{equation} \label{eq:discrete_dynamics}
      x_{k+1} = f(x_k, u_k)
  \end{equation}
  where $x \in \mathcal{X} \subseteq \R^{N_x}$ is the state vector and $u_k \in \R^{N_u}$ is
  the control vector.  This discrete approximation can be obtained for any continuous
  dynamical system in many ways, including implicit and explicit Runge-Kutta methods, or by
  solving the Discrete Euler-Lagrange equations \todo{add citations}.

  The key idea behind the Koopman operator is that the nonlinear finite-dimensional discrete
  dynamics \eqref{eq:discrete_dynamics} can be represented by an infinite-dimensional
  \textit{bilinear} system:
  \begin{equation} \label{eq:bilinear_dynamics}
      y_{k+1} = A y_k + B u_k + \sum_{i=1}^m u_{k,i} C_i y_k = g(y,u)
  \end{equation}
  where $y = \phi(x)$ is a mapping from the finite-dimensional state space $\mathcal{X}$ to
  the (possibly) infinite-dimensional Hilbert space of \textit{observables} $y \in
  \mathcal{Y}$. We also assume the inverse map is approximately linear: $x = G y$. In
  practice, we approximate \eqref{eq:discrete_dynamics} by choosing $\phi$ to be some
  arbitrary finite set of nonlinear functions of the state variables, which in general
  include the states themselves such that the linear mapping $G \in \R^{N_x \times N_y}$ is
  exact.  Intuitively, $\phi$ ``lifts'' our states into a higher dimensional space where the
  dynamics are approximately (bi)linear, effectively trading dimensionality for
  (bi)linearity. This idea should be both unsurprising and familiar to most roboticsts,
  since similar techniques have already been employed in other forms, such as
  maximal-coordinate representations of rigid body dynamics \todo{add citations}, the
  ``kernel trick'' for state-vector machines \todo{add citation}, or the observation that
  solving large, sparse nonlinear optimization problems is often more effective than solving
  small, tightly-coupled dense problems \todo{add citations from the trajectory optimization
  literature}.

  The lifted bilinear system \eqref{eq:bilinear_dynamics} can be easily learned from samples
  of the system dynamics $(x_i^+,x_i,u_i)$ (where $x^+$ is the state at the next time step)
  using extended Dynamic Mode Decomposition (eDMD), which is just the application of linear
  least squares (LLS) to the lifted states.  Details of this method will be covered later
  when we introduce our adaptation of eDMD and present an effective numerical technique for
  solving the resulting LLS problems.

%%%%%%%%%%%%%%%%%%%%%%%%%%%%%%%%%%%%%%%%%%%%%%%%%%%%%%%%%%%%%%%%%%%%%%%%%%%%%%%%%%%%%%%%%%
% Methodology
%%%%%%%%%%%%%%%%%%%%%%%%%%%%%%%%%%%%%%%%%%%%%%%%%%%%%%%%%%%%%%%%%%%%%%%%%%%%%%%%%%%%%%%%%%
\section{EDMD with Jacobian Residual-Penalization} \label{sec:methodology}
  Existing Koopman-based approaches to learning dynamical systems only rely on samples of
  the unknown dynamics. Here we present a novel method for incorporating prior knowledge
  about the dynamics by adding derivative information of an approximate model into the data
  set to be learned via eDMD.

  Given $P$ samples of the dynamics $(x_i^+, x_i, u_i)$, and an approximate discrete
  dynamics model 
  \begin{equation}
      x^+ = \tilde{f}(x,u)
  \end{equation}
  we can evaluate the Jacobians of our approximate model $\hat{f}$ at each of the sample
  points: $\tilde{A}_i = \pdv{\tilde{f}}{x}, \tilde{B}_i = \pdv{\tilde{f}}{x}$. After
  choosing nonlinear mapping $\phi : \R^{N_x} \mapsto \R^{N_y}$ We then want to find a
  bilinear dynamics model \eqref{eq:bilinear_dynamics} that matches the Jacobians of our
  approximate model, while also matching our dynamics samples. If we define $\hat{A}_j \in
  \R^{N_x \times N_x}$ and $\hat{B}_j \in \R^{N_x \times N_u}$ to be the Jacobians of our
  bilinear dynamics model, projected back into the original state space, our objective is to
  find the matrices $A \in \R^{N_y \times N_y},B \in \R^{N_y \times N_u}$, and $C_{1:m} \in
  \R^{N_u} \times \R^{N_y \times N_y}$ that minimize the following objective:

  \begin{equation} \label{eq:lls_objective}
      (1- \alpha) \sum_{j=1}^P \norm{\hat{y}_j - y_j^+}_2^2 + 
          \alpha  \sum_{j=1}^P \norm{\hat{A}_j - \tilde{A}_j}_2^2 + 
                               \norm{\hat{B}_j - \tilde{B}_j}_2^2 
  \end{equation}
  where $\hat{y}_j^+ = g\left(\phi(x_j), u_j\right)$ is the output of our bilinear  dynamics
  model, and $y_j^+ = \phi(y_j^+)$ is the actual lifted state (i.e. observables) at the next
  time step.

  While not immediately apparent, we can minimize \eqref{eq:lls_objective} using linear
  least-squares, using techniques similar to those used previously in the literature
  \todo{cite CalTech papers}.

  To start, we combine all the data we're trying to learn in a single matrix:
  \begin{equation}
      E = \begin{bmatrix} A & B & C_1 & \dots & C_m \end{bmatrix} \in \R^{N_y \times N_z} \\
  \end{equation}
  where $N_z = N_y + N_u + N_y \cdot N_u$.  We now rewrite the terms in
  \eqref{eq:lls_objective} in terms of $E$. By defining the vector 
  \begin{equation}
      z = \begin{bmatrix} y^T & u^T & u_1 y^T & \dots & u_m y^T \end{bmatrix} \in \R^{N_z} 
  \end{equation}
  we can write down 
  the output of our bilinear dynamics \eqref{eq:bilinear_dynamics} as 
  \begin{equation} \label{eq:bilinear_dynamics_z}
      \hat{y}^+ = E z.
  \end{equation}
  The projected Jacobians of our bilinear model, $\hat{A}$ and $\hat{B}$, are simply the Jacobians 
  of the bilinear dynamics in terms of the original state. We obtain these dynamics by ``lifting`` 
  the state via $\phi$ and then projecting back onto the original states using $G$:
  \begin{equation} \label{eq:projected_dynamics}
      x^+ = G \left( A \phi(x) + B u + \sum_{i=1}^m u_i C_i \phi(x) \right)  = \hat{f}(x,u) 
  \end{equation}
  Differentiating these dynamics give us our projected Jacobians:
  \begin{subequations} \label{eq:projected_jacobians}
  \begin{align}
      \hat{A}_j &= G \pdv{\hat{f}}{x}\left(x_j,u_j\right) 
                = G \left(A + \sum_{i=1}^m u_{j,i} C_i \right) \Phi(x_j)
              %   = G A_j^x \phi(x_j)
                = G E \bar{A}(x_j,u_j) = G E \bar{A}_j \\
      \hat{B}_i &= G \pdv{\hat{f}}{u}\left(x_j,u_j\right) 
                = G \Big(B + \begin{bmatrix} C_1 x_j & \dots & C_m x_j \end{bmatrix} \Big)
              %   = G B_j^u
                = G E \bar{B}(x_j,u_j) = G E \bar{B}_j
  \end{align}
  \end{subequations}
  where $\Phi(x) = \pdv*{\phi}{x}$ is the Jacobian of the nonlinear map $\phi$, and
  \begin{equation}
      \bar{A}(x,u) =  \begin{bmatrix} 
          I \\ 0 \\ u_1 I \\ u_2 I \\ \vdots \\ u_m I 
      \end{bmatrix} \in \R^{N_z \times N_x}, \quad
      \bar{B}(x,u) = \begin{bmatrix} 
          0 \\ 
          I \\ 
          [x \; 0 \; ... \; 0] \\
          [0 \; x \; ... \; 0] \\
          \vdots \\
          [0 \; 0 \; ... \; x] \\
      \end{bmatrix} \in \R^{N_z \times N_u}.
  \end{equation}
  Substituting \eqref{eq:bilinear_dynamics_z} and \eqref{eq:projected_jacobians} into
  \eqref{eq:lls_objective}, we can rewrite our LLS problem as:
  \begin{align}
      \underset{E}{\text{minimize}} \;\; 
          \sum_{j=0}^P
          (1-\alpha) \norm{E z_j - y_j^+}_2^2 + 
            \alpha  \norm{G E \bar{A}_j - \tilde{A}_j}_2^2 + 
            \alpha  \norm{G E \bar{B}_j - \tilde{B}_j}_2^2 
  \end{align}
  which is equivalent to
  \begin{align} \label{opt:lls_matrix}
      \underset{E}{\text{minimize}} \;\; 
          (1-\alpha) \norm{E \mathbf{Z_{1:P}} - \mathbf{Y^+_{1:P}} }_2^2 + 
            \alpha  \norm{G E \mathbf{\bar{A}_{1:P}} - \mathbf{\tilde{A}_{1:P}}}_2^2 + 
            \alpha  \norm{G E \mathbf{\bar{B}_{1:P}} - \mathbf{\tilde{B}_{1:P}}}_2^2
  \end{align}
  where $\mathbf{Z_{1:P}} \in \R^{N_z \times P} = [z_1 \; z_2 \; ... \; z_P]$ horizontally
  concatenates all of the samples (equivalent definition for 
  $\mathbf{Y^+_{1:P}} \in \R^{N_y \times P}$, 
  $\mathbf{\bar{A}_{1:P}} \in \R^{N_z \times N_x \cdot P}$, 
  $\mathbf{\tilde{A}_{1:P}} \in \R^{N_z \times N_x \cdot P}$,
  $\mathbf{\bar{B}_{1:P}} \in \R^{N_z \times N_u \cdot P}$, and 
  $\mathbf{\tilde{B}_{1:P}} \in \R^{N_z \times N_u \cdot P}$ ).

  We can rewrite \eqref{opt:lls_matrix} in standard form using the ``vec trick''
  \begin{equation} \label{eq:vectrick}
      \text{vec}(A X B) = (B^T \otimes A) \text{vec}(X)
  \end{equation}
  where $\text{vec}(A)$ stacks the columns of $A$ into a single vector.

  Setting $E$ in \eqref{opt:lls_matrix} equal to $X$ in \eqref{eq:vectrick}, we get
  \begin{align} \label{opt:lls_matrix}
      \underset{E}{\text{minimize}} \;\;  
      \norm{
          \begin{bmatrix}
              (\mathbf{Z_{1:P}})^T \otimes I_{N_y} \\
              (\mathbf{\bar{A}_{1:P}})^T \otimes G \\
              (\mathbf{\bar{G}_{1:P}})^T \otimes G \\
          \end{bmatrix}
          \text{vec}(E)
          +
          \begin{bmatrix}
              \text{vec}(\mathbf{Y^+_{1:P}}) \\
              \text{vec}(\mathbf{\tilde{A}_{1:P}}) \\
              \text{vec}(\mathbf{\tilde{G}_{1:P}})
          \end{bmatrix}
      }_2^2
  \end{align}
  such that the matrix of cofficients has $(N_y + N_x^2 + N_x \cdot N_u) \cdot P$ rows and 
  $N_y \cdot N_z$ columns.

%%%%%%%%%%%%%%%%%%%%%%%%%%%%%%%%%%%%%%%%%%%%%%%%%%%%%%%%%%%%%%%%%%%%%%%%%%%%%%%%%%%%%%%%%%
% Results
%%%%%%%%%%%%%%%%%%%%%%%%%%%%%%%%%%%%%%%%%%%%%%%%%%%%%%%%%%%%%%%%%%%%%%%%%%%%%%%%%%%%%%%%%%
\section{Results} \label{sec:results}
All continuous dynamics were discretized with an explicit fourth-order Runge Kutta 
integrator.

\subsection{Cartpole}
As a simple benchmark example, we use the canonical cartpole system. In lieu of an actual 
hardware experiment (left for future work), we specify two separate analytical models used 
in simulation: the \textit{nominal} model is a simple cartpole system without friction or 
damping, whereas the \textit{simulated} model, used exclusively for simulating the system
in place of a real hardware platform, includes viscous damping on both degrees of freedom,
a tanh Coloumb friction model for the cart, and a deadband on the control signal. 
Additionally, we altered the mass of the cart and pole from the nominal model by 20\% and 
25\%, respectively. See the provided code for the actual values and models used in the 
experiments.


\subsubsection{Training}
To train the bilinear model, we generated 50 training swing-up trajectories using ALTRO, an
open-source nonlinear trajectory optimization solver \todo{add ALTRO citations}, on the 
nominal cartpole model. Each trajectory was 5 seconds long and sampled at 20 Hz.  A linear
MPC controller was then used to track the swing-up trajectories on the simulated system,
resulting in significant tracking error due to the model mismatch. The bilinear eDMD model
was training using the same approach in \cite{Folkestad2021}. For the proposed jDMD method,
the Jacobians of the nominal model were calculated at each of the sample points and the 
bilinear model was learned using the approach outlined in Section \ref{sec:methodology}.

After learning both models, a linear MPC policy was used to track the original swing-up
trajectory generated by ALTRO. The MPC policy linearized the dynamics about the reference 
trajectory, used a quadratic penalty on deviations from the reference, and was solved using 
Riccati recursion for a horizon of 41 time steps (2 seconds). To successfully stability the 
system at the upward equilibrium, the MPC policy extrapolated the terminal state and control
reference indefinitely. For each of the bilinear models, the Jacobians were projected back 
onto space of the original dynamics, resulting in an MPC policy that was just as efficient
to solve online as the nominal model (all controllers ran at several thousand Hertz without
much performance tuning). 

\begin{figure}[t]
  \centering
  \includegraphics[width=0.9\textwidth, height=5cm]{../images/cartpole_mpc_test_error.tikz}
  \label{fig:cartpole_mpc_test_error}
  \caption{MPC tracking error vs training samples for the cartpole.}
\end{figure}

\subsubsection{MPC Tracking Performance}
The tracking error, defined as the average L2 norm of the error between the reference 
trajectory and the actual trajectory of the simulated system for 10 test trajectories not 
included in the training data, was recorded after training both the eDMD and jDMD models
with an increasing number of trajectories. The results in Figure
\ref{fig:cartpole_mpc_test_error} clearly show that jDMD produces a high-quality model with
much fewer samples than eDMD. With just 6 training trajectories, jDMD shows dramatically
improved tracking performance over both the nominal analytical model and the eDMD model. The
slight decrease in tracking performance with increasing samples for jDMD could be due to 
overfitting, lack of expressiveness in the basis functions, or poor tuning of the $\alpha$ 
weighting parameter between the Jacobians and the sample data (which was held at 0.5 for all
examples).



%%%%%%%%%%%%%%%%%%%%%%%%%%%%%%%%%%%%%%%%%%%%%%%%%%%%%%%%%%%%%%%%%%%%%%%%%%%%%%%%%%%%%%%%%%
% Limitations 
%%%%%%%%%%%%%%%%%%%%%%%%%%%%%%%%%%%%%%%%%%%%%%%%%%%%%%%%%%%%%%%%%%%%%%%%%%%%%%%%%%%%%%%%%%

%%%%%%%%%%%%%%%%%%%%%%%%%%%%%%%%%%%%%%%%%%%%%%%%%%%%%%%%%%%%%%%%%%%%%%%%%%%%%%%%%%%%%%%%%%
% Conclusion 
%%%%%%%%%%%%%%%%%%%%%%%%%%%%%%%%%%%%%%%%%%%%%%%%%%%%%%%%%%%%%%%%%%%%%%%%%%%%%%%%%%%%%%%%%%

\bibliography{references.bib}

\end{document}