\documentclass{article}
% \usepackage{corl_2022} % Use this for the initial submission.
\usepackage[final]{corl_2022} % Uncomment for the camera-ready ``final'' version.
\usepackage{times} % assumes new font selection scheme installed
\usepackage{brian}
\graphicspath{{../images/}}
\usepackage[export]{adjustbox}
% \usepackage{biblatex}

% Bibliography
% \usepackage[style=ieee,doi=false,isbn=false,url=false,eprint=false]{biblatex}
% \AtEveryBibitem{\clearlist{language}}  % remove language field from bib entries
% \addbibresource{references.bib}
% \renewcommand*{\bibfont}{\small}


\title{JDMD: Extended Dynamic Mode Decomposition with Jacobian Residual Penalization
for Learning Bilinear, Control-affine Koopman Models}

\author{
  Brian E. Jackson \\
  Robotics Institute \\
  Carnegie Mellon University\\
  \texttt{brianjackson@cmu.edu} \\
  \and
  Jeong Hun Lee \\
  Robotics Institute\\
  Carnegie Mellon University\\
  \texttt{jeonghunlee@cmu.edu} \\
  \and
  Kevin Tracy \\
  Robotics Institute\\
  Carnegie Mellon University\\
  \texttt{ktracy@cmu.edu} \\
  \and
  Zachary Manchester \\
  Robotics Institute\\
  Carnegie Mellon University\\
  \texttt{zacm@cmu.edu} \\
}

\begin{document}
\maketitle

\todo{Add abstract}

\section{Introduction}

    Controlling complex, underactuated, and highly nonlinear autonomous systems remains an
    active area of research in robotics, despite decades of previous work exploring
    effective algorithms and the development of substantial theoretical analysis. Classical
    approaches typically rely on local linear approximations of the nonlinear system, which
    are then used in any of a multitude of linear control techniques, such as PID, pole
    placement, Bode analysis, H-infinity, LQR, or linear MPC.  These approaches only work
    well if the states of the system always remain close to the equilibrium point or
    reference trajectory about which the system was linearized. The region for which these
    linearizations remain valid can be extremely small for highly nonlinear systems.
    Alternatively, model-based methods for nonlinear optimal control have shown great
    success, as long as the model is well known and an accurate estimate of the global state
    can be provided. These model-based techniques leverage decades of insight into
    dynamical systems and have demonstrated incredible performance on complicated
    autonomous systems 
    \cite{farshidian_efficient_2017,Kuindersma2014,Bjelonic2021,Subosits2019} .  On the
    other hand, data-driven techniques such as reinforcement learning have received
    tremendous attention over the last decade and have begun to demonstrate impressive
    performance and robustness for complicated robotic systems in unstructured environments
    \cite{Karnchanachari2020,Hoeller2020,Li2021}. While these approaches are attractive
    since they require little to no previous knowledge about the system, they often require
    large amounts of data and fail to generalize outside of the domain or task on which they
    were ``trained.''
    
    In this work we propose a novel method that combines the benefits of model-based and
    data-driven methods, based on recent work  applying Koopman Operator Theory to
    controlled dynamical systems 
    \cite{Meduri2022,Bruder2021,Korda2018,Folkestad2020,Suh2020}.
    % \cite{Meduri2022,Bruder2021, Korda2021}
    % \cite{Meduri2022, Folkestad2021, Bruder2021, Korda2018, Folkestad2020a, Folkestad2020b}.  
    By leveraging data collected from an
    unknown dynamical system along with derivative information from an approximate
    analytical model, we can efficiently learn a bilinear representation of the system
    dynamics that performs well when used in traditional model-based control techniques such
    as linear MPC. By leveraging information from an analytical model, we can dramatically
    reduce the number of samples required to learn a good approximation of the true
    nonlinear dynamics. We also show the effiectiveness of linear MPC on these systems 
    when the learned bilinear system is linearized and projected back into the original 
    state space. The result is a fast, robust, and sample-efficient pipeline for quickly 
    learning a model that beats previous Koopman-based linear MPC approaches as well as 
    purely model-based linear MPC contollers that do not leverage data collected from the 
    actual system. To efficiently learn these bilinear representations, we also propose 
    a numerical technique that allows for large systems to be trained while limiting the 
    peak memory required to solve the least-squares problem.

    In summary, our contributions are:
    \begin{itemize}
        \item A novel extension to extended dynamic mode decomposition (eDMD) that
        incorporates gradient information from an approximate analytic model;

        \item a simple linear MPC control technique for learned bilinear control systems
        that is computationally efficient online, which, when combined with the proposed
        extension to eDMD, requires extremely little training data to get a good control
        policy; and

        \item a recursive, batch QR algorithm solving the least-squares problems that arise 
        when learning bilinear dynamical systems using eDMD.
    \end{itemize}
    
    The paper is organized as follows: in Section \ref{sec:Preliminaries/Background} we 
    give some background on the appliciation of Koopman operator theory to controlled 
    dynamical systems and review some related works. Section \ref{sec:methodology} describes
    the proposed algorithm for combining data-driven and model-based approaches, along with 
    the numerical method for solving the resulting large and sparse linear least-squares 
    problems. Section \ref{sec:results} provides extensive numerical analysis of the 
    proposed algorithm, applied to a simulated cartpole and planar quadrotor model, both 
    subject to significant model mistmach. In Section \ref{sec:limitations} we discuss the 
    limitations of the method, and finish with some concluding thoughts in Section 
    \ref{sec:conclusion}.

\section{Background and Related Work} \label{sec:Preliminaries/Background}

  \subsection{The Koopman Operator}
  The theoretical underpinnings of the Koopman operator and its application to dynamical
  systems has been extensively studied, especially within the last decade 
  \cite{Fasel2021,Proctor2018,Bruder2021,Williams2015}. Rather than describe the theory in
  detail, we highlight the key concepts employed by the current work, and defer the
  motivated reader to the existing literature on Koopman theory.

  We start by assuming we have some discrete approximation a of controlled nonlinear,
  time-dynamical system whose underlying continuous dynamics are Lipschitz continuous:
  \begin{equation} \label{eq:discrete_dynamics} 
    x^+ = f(x, u) 
  \end{equation} 
  where $x \in \mathcal{X} \subseteq \R^{N_x}$ is the state vector, $u_k \in \R^{N_u}$ is
  the control vector, and $x^+$ is the state at the next time step. This discrete
  approximation can be obtained for any continuous-time, smooth dynamical system in many
  ways, including implicit and explicit Runge-Kutta methods, or by solving the Discrete
  Euler-Lagrange equations \cite{Brudigam2021a,Brudigam2021,Howell2022}.

  The key idea behind the Koopman operator is that the nonlinear finite-dimensional discrete
  dynamics \eqref{eq:discrete_dynamics} can be represented by an infinite-dimensional
  \textit{bilinear} system:
  \begin{equation} \label{eq:bilinear_dynamics}
      y^+ = A y + B u + \sum_{i=1}^m u_i C_i y = g(y,u)
  \end{equation}
  where $y = \phi(x)$ is a nonlinear mapping from the finite-dimensional state space
  $\mathcal{X}$ to the (possibly) infinite-dimensional Hilbert space of \textit{observables}
  $y \in \mathcal{Y}$. We also assume the inverse map is approximately linear: $x = G y$. In
  practice, we approximate \eqref{eq:discrete_dynamics} by choosing $\phi$ to be some
  arbitrary finite set of nonlinear functions of the state variables, which in general
  include the states themselves such that the linear mapping $G \in \R^{N_x \times N_y}$ is
  exact.  Intuitively, $\phi$ ``lifts'' our states into a higher dimensional space where the
  dynamics are approximately (bi)linear, effectively trading dimensionality for
  (bi)linearity. This idea should be both unsurprising and familiar to most roboticsts,
  since similar techniques have already been employed in other forms, such as
  maximal-coordinate representations of rigid body dynamics
  \cite{baraff_linear-time_1996-1,Brudigam2021a,Howell2022}, the
  ``kernel trick'' for state-vector machines \cite{Hofmann2006}, or the observation that
  solving large, sparse nonlinear optimization problems is often more effective than solving
  small, tightly-coupled dense problems \todo{add citations from the trajectory optimization
  literature}.

  The lifted bilinear system \eqref{eq:bilinear_dynamics} can be easily learned from samples
  of the system dynamics $(x_j^+,x_j,u_j)$ using extended Dynamic Mode Decomposition (eDMD)
  \cite{Williams2015}, which is just the application of linear least squares (LLS) to the
  lifted states. Details of this method will be covered in the next section where we
  introduce our adaptation of eDMD and present an effective numerical technique for solving
  the resulting LLS problems.

  \todo{mention the most related papers}

%%%%%%%%%%%%%%%%%%%%%%%%%%%%%%%%%%%%%%%%%%%%%%%%%%%%%%%%%%%%%%%%%%%%%%%%%%%%%%%%%%%%%%%%%%
% Methodology
%%%%%%%%%%%%%%%%%%%%%%%%%%%%%%%%%%%%%%%%%%%%%%%%%%%%%%%%%%%%%%%%%%%%%%%%%%%%%%%%%%%%%%%%%%
\section{EDMD with Jacobian Residual-Penalization} \label{sec:methodology}
  Existing Koopman-based approaches to learning dynamical systems only rely on samples of
  the unknown dynamics. Here we present a novel method for incorporating prior knowledge
  about the dynamics by adding derivative information of an approximate model into the data
  set to be learned via eDMD.

  Given $P$ samples of the dynamics $(x_i^+, x_i, u_i)$, and an approximate discrete
  dynamics model 
  \begin{equation}
      x^+ = \tilde{f}(x,u)
  \end{equation}
  we can evaluate the Jacobians of our approximate model $\tilde{f}$ at each of the sample
  points: $\tilde{A}_i = \pdv{\tilde{f}}{x}, \tilde{B}_i = \pdv{\tilde{f}}{u}$. After
  choosing a nonlinear mapping $\phi : \R^{N_x} \mapsto \R^{N_y}$ our goal is to find a
  bilinear dynamics model \eqref{eq:bilinear_dynamics} that matches the Jacobians of our
  approximate model, while also matching our dynamics samples. If we define $\hat{A}_j \in
  \R^{N_x \times N_x}$ and $\hat{B}_j \in \R^{N_x \times N_u}$ to be the Jacobians of our
  bilinear dynamics model, projected back into the original state space (a formal definition
  of these terms will be provided in a few paragraphs), our objective is to find the
  matrices parameterizing our bilinear dynamics model, $A \in \R^{N_y \times N_y},B \in
  \R^{N_y \times N_u}$, and $C_{1:m} \in \R^{N_u} \times \R^{N_y \times N_y}$, that minimize
  the following objective:

  \begin{equation} \label{eq:lls_objective}
      (1- \alpha) \sum_{j=1}^P \norm{\hat{y}_j - y_j^+}_2^2 + 
          \alpha  \sum_{j=1}^P \norm{\hat{A}_j - \tilde{A}_j}_2^2 + 
                               \norm{\hat{B}_j - \tilde{B}_j}_2^2 
  \end{equation}
  where $\hat{y}_j^+ = g\left(\phi(x_j), u_j\right)$ is the output of our bilinear dynamics
  model, and $y_j^+ = \phi(x_j^+)$ is the actual lifted state (i.e. observables) at the next
  time step. Note that $\hat{y}_j$, $\hat{A}_j$, and $\hat{B}_j$ are all implicitly
  functions of the model parameters $A$, $B$, and $C_{1:m}$ we're trying to learn.

  While not immediately apparent, we can minimize \eqref{eq:lls_objective} using linear
  least-squares, using techniques similar to those used previously in the literature
  \cite{Folkestad2021}.

  To start, we combine all the data we're trying to learn into a single matrix:
  \begin{equation}
      E = \begin{bmatrix} A & B & C_1 & \dots & C_m \end{bmatrix} \in \R^{N_y \times N_z},
  \end{equation}
  where $N_z = N_y + N_u + N_y \cdot N_u$.  We now rewrite the terms in
  \eqref{eq:lls_objective} in terms of $E$. By defining the vector 
  \begin{equation}
      z = \begin{bmatrix} y^T & u^T & u_1 y^T & \dots & u_m y^T \end{bmatrix} \in \R^{N_z},
  \end{equation}
  we can write down 
  the output of our bilinear dynamics \eqref{eq:bilinear_dynamics} as 
  \begin{equation} \label{eq:bilinear_dynamics_z}
      \hat{y}^+ = E z.
  \end{equation}
  The previously-mentioned projected Jacobians of our bilinear model, $\hat{A}$ and
  $\hat{B}$, are simply the Jacobians of the bilinear dynamics in terms of the original
  state. We obtain these dynamics by ``lifting`` the state via $\phi$ and then projecting
  back onto the original states using $G$:
  \begin{equation} \label{eq:projected_dynamics}
      x^+ = G \left( A \phi(x) + B u + \sum_{i=1}^m u_i C_i \phi(x) \right)  = \hat{f}(x,u) 
  \end{equation}
  Differentiating these dynamics gives us our projected Jacobians:
  \begin{subequations} \label{eq:projected_jacobians}
  \begin{align}
      \hat{A}_j &= \pdv{\hat{f}}{x}\left(x_j,u_j\right) 
                = G \left(A + \sum_{i=1}^m u_{j,i} C_i \right) \Phi(x_j)
              %   = G A_j^x \phi(x_j)
                = G E \bar{A}(x_j,u_j) = G E \bar{A}_j \\
      \hat{B}_j &= \pdv{\hat{f}}{u}\left(x_j,u_j\right) 
                = G \Big(B + \begin{bmatrix} C_1 x_j & \dots & C_m x_j \end{bmatrix} \Big)
              %   = G B_j^u
                = G E \bar{B}(x_j,u_j) = G E \bar{B}_j
  \end{align}
  \end{subequations}
  where $\Phi(x) = \pdv*{\phi}{x}$ is the Jacobian of the nonlinear map $\phi$, and
  \begin{equation}
      \bar{A}(x,u) =  \begin{bmatrix} 
          I \\ 0 \\ u_1 I \\ u_2 I \\ \vdots \\ u_m I 
      \end{bmatrix} \in \R^{N_z \times N_x}, \quad
      \bar{B}(x,u) = \begin{bmatrix} 
          0 \\ 
          I \\ 
          [x \; 0 \; ... \; 0] \\
          [0 \; x \; ... \; 0] \\
          \vdots \\
          [0 \; 0 \; ... \; x] \\
      \end{bmatrix} \in \R^{N_z \times N_u}.
  \end{equation}
  Note we define $\bar{A}_j = \bar{A}(x_j,u_j)$, $\bar{B}_j = \bar{B}(x_j,u_j)$ to lighten 
  the notation, but want to emphasize that these terms are all purely functions of the input
  data.

  Substituting \eqref{eq:bilinear_dynamics_z} and \eqref{eq:projected_jacobians} into
  \eqref{eq:lls_objective}, we can rewrite our LLS problem as:
  \begin{align}
      \underset{E}{\text{minimize}} \;\; 
          \sum_{j=0}^P
          (1-\alpha) \norm{E z_j - y_j^+}_2^2 + 
            \alpha  \norm{G E \bar{A}_j - \tilde{A}_j}_2^2 + 
            \alpha  \norm{G E \bar{B}_j - \tilde{B}_j}_2^2 
  \end{align}
  which is equivalent to
  \begin{align} \label{opt:lls_matrices}
      \underset{E}{\text{minimize}} \;\; 
          (1-\alpha) \norm{E \mathbf{Z_{1:P}} - \mathbf{Y^+_{1:P}} }_2^2 + 
            \alpha  \norm{G E \mathbf{\bar{A}_{1:P}} - \mathbf{\tilde{A}_{1:P}}}_2^2 + 
            \alpha  \norm{G E \mathbf{\bar{B}_{1:P}} - \mathbf{\tilde{B}_{1:P}}}_2^2
  \end{align}
  where $\mathbf{Z_{1:P}} \in \R^{N_z \times P} = [z_1 \; z_2 \; ... \; z_P]$ horizontally
  concatenates all of the samples (equivalent definition for 
  $\mathbf{Y^+_{1:P}} \in \R^{N_y \times P}$, 
  $\mathbf{\bar{A}_{1:P}} \in \R^{N_z \times N_x \cdot P}$, 
  $\mathbf{\tilde{A}_{1:P}} \in \R^{N_z \times N_x \cdot P}$,
  $\mathbf{\bar{B}_{1:P}} \in \R^{N_z \times N_u \cdot P}$, and 
  $\mathbf{\tilde{B}_{1:P}} \in \R^{N_z \times N_u \cdot P}$ ).

  We can rewrite \eqref{opt:lls_matrices} in standard form using the ``vec trick''
  \begin{equation} \label{eq:vectrick}
      \text{vec}(A X B) = (B^T \otimes A) \text{vec}(X)
  \end{equation}
  where $\text{vec}(A)$ stacks the columns of $A$ into a single vector.

  Setting $E$ in \eqref{opt:lls_matrix} equal to $X$ in \eqref{eq:vectrick}, we get
  \begin{align} \label{opt:lls_matrix}
      \underset{E}{\text{minimize}} \;\;  
      \norm{
          \begin{bmatrix}
              (1-\alpha)\cdot(\mathbf{Z_{1:P}})^T \otimes I_{N_y} \\
              \alpha\cdot(\mathbf{\bar{A}_{1:P}})^T \otimes G \\
              \alpha\cdot(\mathbf{\bar{G}_{1:P}})^T \otimes G \\
          \end{bmatrix}
          \text{vec}(E)
          +
          \begin{bmatrix}
              (1-\alpha)\cdot\text{vec}(\mathbf{Y^+_{1:P}}) \\
              \alpha\cdot\text{vec}(\mathbf{\tilde{A}_{1:P}}) \\
              \alpha\cdot\text{vec}(\mathbf{\tilde{G}_{1:P}})
          \end{bmatrix}
      }_2^2
  \end{align}
  such that the matrix of cofficients has $(N_y + N_x^2 + N_x \cdot N_u) \cdot P$ rows and 
  $N_y \cdot N_z$ columns. We obtain the data for our bilinear model 
  \eqref{eq:bilinear_dynamics} by solving this large, sparse linear least-squares 
  problem.

\subsection{Efficient Recursive Least Squares}
In its canonical formulation, a linear least squares problem can be represented as the
following unconstrained optimization problem:
\begin{align}
    \min_x \|Fx - d\|_2^2.
\end{align}
The solution to this problem is found by solving for the $x$ in which the gradient of the
objective function with respect to $x$ is zero, also known as the normal equations: 
\begin{align}\label{eq:normal_eq}
    (F^TF)x =F^Td,
\end{align}
For small to medium sized problems, this problem is most often solved with either a Cholesky
or a QR decomposition.  Unfortunately, for very large problems where storage size and
numerical conditioning become a concern, forming and factorizing the required matrices can
be intractable.

To deal with large problems like the one proposed in \eqref{opt:lls_matrix}, a recursive
method is used that processes rows of $F$ and $d$ sequentially in batches, avoiding the need
for forming or factorizing the whole matrix. To do this, the rows of $F$ and $d$ will be
divided up into batches in the following manner:
\begin{equation}
    \begin{aligned}
        F = \begin{bmatrix} F_1 \\ F_2 \\ \vdots \\ F_N \end{bmatrix}
    \end{aligned},
    \quad 
    \begin{aligned}
        d = \begin{bmatrix} d_1 \\ d_2 \\ \vdots \\ d_N \end{bmatrix}.
    \end{aligned}
\end{equation}
The matrix $F^TF$ from \eqref{eq:normal_eq} can then be represented as the following sum:
\begin{align} \label{eq:F_sum}
    F^TF = F_1^TF_1 + F_2^TF_2 + \ldots + F_N^TF,
\end{align}
with the right-hand side vector in \eqref{eq:normal_eq} expressed in a similar fashion:
\begin{align}
    F^Td = F_1^Td_1 + F_2^Td_2 + \ldots + F_N^Td_N.
\end{align}
The main idea of this recursive method is to maintain an upper-triangular ``square root''
factor $U_i$ of the first $i$ terms of the sum \eqref{eq:F_sum}. Given the factorization 
$U_i$, we can calculate $U_{k+1}$ using the $\operatorname{QR}$ decomposition, as shown in
\cite{Howell2019}:
\begin{equation}
    U_{i+1} = \sqrt{U_i + F_{i+1}} = 
    \operatorname{QR_R}\bigg( \begin{bmatrix} \sqrt{U_i} \\ \sqrt{F_{i+1}} \end{bmatrix} \bigg),
\end{equation}
where $\operatorname{QR_R}$ returns the upper triangular matrix $R$ from the 
$\operatorname{QR}$ decomposition. 

We handle regularization of the normal equations, equivalent to adding L2 regularization to 
the original least squares problem, during the base case of our recursion. If we want to 
add an L2 regularization with weight $\rho$, we calculate $U_1$ as:

\begin{equation}
    U_1 =  \operatorname{QR_R}\bigg( 
        \begin{bmatrix} \sqrt{F_1} \\ \sqrt{\rho} I \end{bmatrix}.
    \bigg),
\end{equation}

The final algorithm for solving a least squares problem in a recursive-batch fashion is
described in algorithm \ref{alg:rlsqr}. This algorithm can be modified to handle L2
regularized least squares problems by simply replacing line 2 of algorithm \ref{alg:rlsqr}
with $U\leftarrow \operatorname{QR_R}(\operatorname{vcat}(F_1,\sqrt{\rho}I))$, where $\rho$
is the regularizer. 
 \begin{algorithm} 
    \begin{algorithmic}[1]
        \caption{Recursive Batch Least Squares with QR}\label{alg:rlsqr}
        \State \textbf{input} $F,\,d$  \Comment{problem data}
        \State $U \leftarrow \operatorname{QR_R}(\operatorname{vcat}(F_1,\sqrt{\rho}I))$ \Comment{form initial upper-triangular square-root}
        \State $b \leftarrow F_1^Td_1$ \Comment{form initial right hand side vector}
        \For{$i = 2:N$}
% 		\State $U \leftarrow \operatorname{QR_R}([U^T \, A_i^T]^T) $ 
        \State $U \leftarrow \operatorname{QR_R}(\operatorname{vcat}(U,F_i)) $ \Comment{update square-root with new batch}
        \State $b \leftarrow b + F_i^Td_i$ \Comment{update right hand side with new batch}
        \EndFor
        \State \textbf{ouput} \,$x \leftarrow U^{-1}U^{-T}b$ \Comment{forward and backwards substitution to solve for $x$}
    \end{algorithmic}
\end{algorithm}

%%%%%%%%%%%%%%%%%%%%%%%%%%%%%%%%%%%%%%%%%%%%%%%%%%%%%%%%%%%%%%%%%%%%%%%%%%%%%%%%%%%%%%%%%%
% Results
%%%%%%%%%%%%%%%%%%%%%%%%%%%%%%%%%%%%%%%%%%%%%%%%%%%%%%%%%%%%%%%%%%%%%%%%%%%%%%%%%%%%%%%%%%
\section{Results} \label{sec:results}
The following sections provide various numerical analyses of the proposed algorithm. 
In lieu of an actual hardware experiment (left for future work), for each simulated system 
we specify two models: a \textit{nominal} model which is a simplified model approximating 
the \textit{simulated} model, which contains both parametric and non-parametric model 
error from the nominal model, and is used exclusively for simulating the system.

All models were trained by simulating the ``real'' system with an arbitrary controller to 
collect data in the region of the state space relevant to the task. A set of fixed-length 
trajectories were collected, each at a sample rate of 20 Hz. The bilinear eDMD model was
trained using the same approach in \cite{Folkestad2021}. For the proposed jDMD method, the
Jacobians of the nominal model were calculated at each of the sample points and the bilinear
model was learned using the approach outlined in Section \ref{sec:methodology}.
All continuous dynamics were discretized with an explicit fourth-order Runge Kutta 
integrator. The code for the experiments is located at 
\todo{include after review}.

\subsection{Sample Efficiency}

We highlight the sample efficiency of the proposed algorithm in Figures 
\ref{fig:cartpole_mpc_test_error} and \ref{fig:rex_planar_quadrotor_mpc_test_error}. For
both the cartpole swingup and the quadrotor trajectory tracking tasks, the proposed method
achieves better tracking than the nominal MPC controller after just a few training
trajectories. In both cases, the classic eDMD approach never achieves the same level of
performance as the proposed approach. The lack of continued progress with increasing samples
is likely due to a lack of sufficient variety in the training data: after 30-40 training
trajectories both methods have effectively learned as much as they can from the distribution
from which the training data was sampled.

The training time versus number of training samples is shown in Figure 
\ref{fig:cartpole_train_time} for the cartpole swingup task. While the proposed approach 
naturally takes longer since it includes much more information per sample (adding 
$N_x + N_u + 1$ rows for every sample), the complexity the approximately linear and the 
solve times are on the order of seconds for simple systems.

\subsection{Lifted versus Projected MPC}
Figure \ref{fig:cartpole_lqr_samples} also highlights the sample efficiency of the proposed 
method, while also comparing to the more typical approach of applying the MPC policy in the 
lifted state space. As shown, the proposed method of applying linear MPC to the projected 
linearization is much more sample efficient than trying to apply it in the lifted state 
space. This approach is also advantegeous because it reduces the solve time of the linear 
MPC policy, is more numerically robust (we found the lifted MPC policies tended to suffer 
from numerical issues when using Riccati recursion to solve for long time horizons), and 
it is straightforward to apply additional constraints to the original state variables.

\subsection{Generalization}
We demonstrate the generalizability of the proposed method to tasks outside of its training
domain in Figures \ref{fig:rex_planar_quadrotor_lqr_error_by_training_window} and 
\ref{fig:rex_planar_quadrotor_mpc_error_by_training_window}. In both the quadrotor 
stabilization (Figure \ref{fig:rex_planar_quadrotor_lqr_error_by_training_window}) and 
trajectory tracking (Figure \ref{fig:rex_planar_quadrotor_mpc_error_by_training_window})
tasks, we trained the models by sampling uniformly from a given window of offsets, centered 
about the origin. 
To test the generalizability of the methods we increased the relative size of the window 
from which the test data was sampled, e.g. if the initial lateral position was trained on 
data in the interval $[-1.5,1.5]$, we sampled the test initial condition from the window 
$[\gamma -1.5, \gamma 1.5]$

\subsection{Cartpole}
As a simple benchmark example, we use the canonical cartpole system. The \textit{simulated}
cartpole model included a $\tanh$ model of Columb friction between the cart and the 
floor, viscous damping at both joints, and a control deadband that was not included in the 
\textit{nominal} model. Additionally the mass of the cart and pole of the simulated model 
were altered by 20\% and 25\% with respect to the nominal model, respectively. 

We split the analysis of this system into two separate tasks: stabilization about the upward
unstable equilibrium from perturbed initial condtions, and the swing-up task where the 
system must successfully stabilize after starting from the downward equilibrium.

\subsubsection{Stabilization}

For stabilizing the system about the upward unstable equilibrium, we used an LQR controller
to collect trajectories on the simulated system. To learn
the bilinear models, we used the following nonlinear mapping: $\phi(x) = [\; 1,\; x,\;
\sin(x),\; \cos(x),\; \sin(2x)\; ] \in \R^{17}$.  After learning both models, an MPC
controller was designed using the nominal, eDMD, and jDMD models. For the learned bilinear
models, MPC controllers were designed using both the ``lifted'' Jacobians of the bilinear 
dynamics and the ``projected'' Jacobians \eqref{eq:projected_jacobians}. The MPC controllers
linearized the dynamics about the equilibrium of $x = [\;0,\; \pi,\; 0,\;, 0\;]$ and solved
the resulting equality-constrained quadratic program using Riccati recursion and a horizon
of 41 time steps (2 seconds). 

To analyze sample efficiency of the algorithms, we trained the bilinear models with an 
increasing number of samples. Each controller was tested using 100 different initial
conditions sampled from a uniform distribution of initial conditions centered about the 
upward equilibirum. The average L2 error of the state after 4 seconds and the upward 
equilibrium was recorded for each controller. The minimum number of training trajectories to
get performance consistently better than the nominal MPC controller is reported in Figure 
\ref{fig:cartpole_lqr_samples}. As shown, the proposed approach is much more sample
efficient even for this relatively simple task than traditional eDMD. It also shows the
benefit of applying the control in the original state space by projecting the linearization
back to the original states. To our knowledge, no previous works on applying DMD or Koopman
operator theory to controlled systems have used this technique of projecting back into the
original state space, although the benefits are immediately apparent: with just a few
training trajectories (1 for jDMD and 18 for eDMD) we can learn a model that improves upon
our nominal LQR controller policy.

% \begin{figure}
%   \centering
%   \begin{subfigure}[b]{0.40\textwidth}
%     \includegraphics[width=\textwidth,height=5cm]{cartpole_lqr_stabilization_performance.tikz}
%     \caption{LQR Controllers}
%   \end{subfigure}
%   \begin{subfigure}[b]{0.59\textwidth}
%     \includegraphics[width=\textwidth,height=5cm]{cartpole_mpc_stabilization_performance.tikz}
%     \caption{MPC Controller}
%   \end{subfigure}
%   \label{fig:cartpole_stabilization}
%   \caption{Controller performance on the task of stabilizing the cartpole system about the 
%   upward unstable equilibrium. Performance is plotted as the error }
% \end{figure}
\begin{figure}
  \centering
  \includegraphics[width=0.7\textwidth, height=4cm]{cartpole_lqr_samples.tikz}
  \caption{Number of training trajectories requires to beat the nominal MPC controller.
    The criteria is the L2 norm of the state from the goal state after 4 seconds.
    The ``Lifted'' MPC controllers compute the MPC solution in the lifted state space 
    (17 states), whereas the ``Projected'' MPC contollers project the Jacobians of the 
    bilinear system back into the original state space.
  }
  \label{fig:cartpole_lqr_samples}
\end{figure}

\begin{figure}[t]
  \centering
  \begin{subfigure}[t]{0.48\textwidth}
    \raggedleft
    \includegraphics[width=\textwidth, height=5cm]{../images/cartpole_mpc_test_error.tikz}
    \caption{MPC tracking error vs training samples for the cartpole. Tracking error is
    defined as the average L2 error over all the test trajectories between the reference and
    simulated trajectories. The proposed jDMD method immediately learns a model good enough
    for control with just a couple hundred dynamics samples (2 swing-up trajectories).}
    \label{fig:cartpole_mpc_test_error}
  \end{subfigure}
  \hfill
  \begin{subfigure}[t]{0.48\textwidth}
    \raggedright
    \includegraphics[width=\textwidth, height=5cm]{cartpole_mpc_train_time.tikz}
    \caption{Training time for cartpole models as a function of training samples. The 
    training time complexity is approximately linear.}
    \label{fig:cartpole_train_time}
  \end{subfigure}
\end{figure}

\subsubsection{MPC Tracking Performance}

To train the bilinear model for the swing-up task, we generated 50 training swing-up
trajectories using ALTRO, an open-source nonlinear trajectory optimization solver 
\cite{Howell2019,Jackson2021}, on the nominal cartpole model. Each trajectory was 5 seconds
long and sampled at 20 Hz.  A linear MPC controller was then used to track the swing-up
trajectories on the simulated system, resulting in significant tracking error due to the
model mismatch.  After learning both models, a linear MPC policy using the projected 
dynamics Jacobians was used to track the original swing-up trajectory generated by ALTRO. 

The tracking error, defined as the average L2 norm of the error between the reference 
trajectory and the actual trajectory of the simulated system for 10 test trajectories not 
included in the training data, was recorded after training both the eDMD and jDMD models
with an increasing number of trajectories. The results in Figure
\ref{fig:cartpole_mpc_test_error} clearly show that jDMD produces a high-quality model with
extremely few samples, whereas eDMD---even after many training samples---never achieves the 
same level of performance. The lack of progress with increasing samples is likely a result 
of poor variety in the training data: after enough samples both methods effectively learn 
all the information that can be learned from the distribution from which the training data 
is sampled. This example highlights the value of adding even just a little derivative 
information to a data-driven approach: it dramatically increases sample efficiency while 
also improving the quality of the learned model, especially when that model is used in 
optimization-based controllers such as MPC that rely on derivative information. For 
reference, the training times are shown in Figure \ref{fig:cartpole_train_time}. Note that 
the training algorithms haven't yet been optimized for max performance but reflect a decent 
first implementation.


\subsection{Planar Quadrotor}
As another benchmark example, we use the planar quadrotor model, a simplification of the
full quadrotor system with only 3 degrees of freedom. The \textit{simulated} model included
aerodynamic drag terms not included in the \textit{nominal}, as well as parametric error of 
about 5\% on the system properties (e.g. mass, rotor arm length, etc.).

Analysis of the planar quadrotor system is also split into 2 separate tasks: stabilization
about the hover position with perturbed initial conditions, and a point-to-point translation
where the system tracks a linear trajectory between two points with zero initial and 
terminal velocity.

\subsubsection{Stabilization}

For stabilizing the system about the upward unstable equilibrium, we used an LQR controller
designed using the nominal planar quadrotor model to collect various trajectories. To learn
our bilinear eDMD and jDMD, the following nonlinear mapping was used: $\phi(x) = [\; 1,\;
x,\; \sin(x),\; \cos(x),\; 2x^{2}-1\; ] \in \R^{25}$. After learning both models, respective
LQR controllers were designed using the bilinear models' ``projected'' Jacobians
\eqref{eq:projected_jacobians}. The LQR controllers linearized about hover at the origin ($x
= [\;0,\; 0,\; 0,\;, 0\;  0\;  0\;]$) and solved the resulting equality-constrained
quadratic program using Riccati recursion. Each controller was tested using 30 different
initial conditions sampled from a uniform distribution of initial conditions centered about
the hover position.

To study how robust eDMD and jDMD are to regularization, we trained multiple bilinear eDMD
and jDMD models with increasing L2 (Tikhonov) regularization values. LQR controllers were
designed for each model and tested with 100 initial conditions sampled from a uniform
distribution centered about the hover position. The average L2 error of the state after 5.0
seconds of simulation, which we'll refer to in this section as the stabilization error, was
recorded for each model. As shown in Figure
\ref{fig:rex_planar_quadrotor_lqr_error_by_reg}, the proposed jDMD method remains robust
over a range of small regularization values ($<10^{-1}$) when compared to nominal eDMD, which 
clearly has a local optimal around $10^0$. This supports what we found empirically, that 
jDMD tends to be much ``easier'' to train and is far more robust to hyperparameters such as 
the regularization value, when compared to eDMD.

As suggested by Figure \ref{fig:rex_planar_quadrotor_lqr_error_by_reg}, eDMD without
regularization performs best, but may be prone to overfitting. Therefore, we test eDMD's and
jDMD's respective LQR controllers on initial conditions sampled from increasingly larger
uniform distributions to analyze the generalizability of the lifted, bilinear models. The
LQR controllers are tested on 100 samples generated from the uniform distribution, and the
average stabilization error is recorded. As seen in Figure
\ref{fig:rex_planar_quadrotor_lqr_error_by_training_window}, jDMD's LQR controller is much
more robust than eDMD, even when the initial conditions go beyond the scope of the training
data. This suggests that adding Jacobian information not only makes eDMD more sample
efficient, but can also decrease overfitting and improve generalization. 

To further study the robustness of eDMD and jDMD, we tasked the respective LQR controllers for eDMD and jDMD to stabilize under increasingly-varying equilibrium positions. 50 equilibrium positions are sampled from a uniform distribution with increasing bounds, which represent the maximum offset from the origin. The samples are used to test the models' LQR controllers before recording the average stabilization error. The results in Figure \ref{fig:rex_planar_quadrotor_lqr_error_by_equilibrium_change} show that jDMD is very robust to the change in equilibrium and is able to successfully stabilize about the equilibrium despite the increasing offset. This is expected because stabilization performance should be invariant with respect to the quadrotor's equilibrium position, and this information is provided by the Jacobians in jDMD. eDMD's increasing inability to stabilize for greater equilibrium offsets is likely a result of eDMD being overfitted to the training trajectories, which all stabilize about the origin. Therefore, this example once again highlights the value of adding Jacobian information: it is not only a data augmentation technique for sample efficiency, but also a regularizer that can decrease overfitting, making the learned models (and subsequent controllers) more robust and generalizable.

\begin{figure}
    \centering
    \includegraphics[width=\textwidth,height=5cm]{rex_planar_quadrotor_lqr_error_by_reg.tikz}
    \caption{LQR stabilization with equilibrium offset}
    \label{fig:rex_planar_quadrotor_lqr_error_by_reg}
    \caption{ LQR stabilization error over a range of L2 regularization values. jDMD is much more robust and consistent over small regularization values.}
\end{figure}

\begin{figure}
    \centering
    \begin{subfigure}[t]{0.45\textwidth}
        \includegraphics[width=\textwidth,height=5cm]{rex_planar_quadrotor_lqr_error_by_equilibrium_change.tikz}
        \caption{LQR stabilization error over increasing equilibrium offset}
        \label{fig:rex_planar_quadrotor_lqr_error_by_equilibrium_change}
    \end{subfigure}
    \begin{subfigure}[t]{0.45\textwidth}
        \includegraphics[width=\textwidth,height=5cm]{rex_planar_quadrotor_lqr_error_by_training_window.tikz}
        \caption{LQR stabilization error over varying range of distribution for sampling initial conditions. A training range fraction greater than $1$ indicates the distribution range is beyond that used to generate the training trajectories}
        \label{fig:rex_planar_quadrotor_lqr_error_by_training_window}
    \end{subfigure}
    \caption{ LQR controller robustness to varying initial conditions and equilibrium offset. jDMD is much more robust overall with an ability to stabilize outside the scope of training data }
\end{figure}

\subsubsection{MPC Tracking Performance}

To train the bilinear model for the MPC tracking task, we generated 50 infeasible point-point
linear trajectories that all ended at the origin with the planar quadrotor in hover. The starting
initial condition of the system was sampled from a uniform distribution about the origin.
A linear MPC controller was then used to track the infeasible linear trajectories on the
simulated system, resulting in significant tracking error due to the model mismatch. 
After learning both models, a linear MPC policy (identical to the cartpole example) was
used to track the original linear trajectory as best as possible.

The tracking error we recorded for 35 test trajectories after training both the eDMD 
and jDMD models with an increasing number of trajectories. Like the cartpole example,
results in Figure \ref{fig:rex_planar_quadrotor_mpc_test_error} clearly show that jDMD
produces a high-quality model for MPC tracking with extremely few samples, whereas eDMD
---even after many training samples---is never able to develop a model with consistent
and similar performance. This is once again highlighted
in the results of Figure \ref{fig:rex_planar_quadrotor_mpc_error_by_training_window}, which
show that eDMD is only able to successfully track trajectories before falling apart when the
range of test trajectories exceeds that of the training data. Meanwhile, jDMD is able to consistently
track the trajectory to the goal state, despite having greater tracking error in a certain range
of trajectories---this further hints at eDMD overfitting on the training data.

\begin{figure}[t]
    \centering
    \begin{subfigure}[t]{0.48\textwidth}
        \raggedleft
        \includegraphics[width=\textwidth, height=5cm]{rex_planar_quadrotor_mpc_test_error.tikz}
        \caption{MPC tracking error vs training samples for the planar quadrotor. The proposed jDMD method again learns a model good enough with ~1000 samples while eDMD is never able to converge to a good-enough model.}
        \label{fig:rex_planar_quadrotor_mpc_test_error}
    \end{subfigure}
    \hfill
    \begin{subfigure}[t]{0.48\textwidth}
        \raggedright
        \includegraphics[width=\textwidth, height=5cm]{rex_planar_quadrotor_mpc_error_by_training_window.tikz}
        \caption{MPC tracking error of quadrotor over varying range of distribution for sampling initial conditions. Drastically increasing error for eDMD suggests greater overfitting.}
        \label{fig:rex_planar_quadrotor_mpc_error_by_training_window}
    \end{subfigure}
\end{figure}

%%%%%%%%%%%%%%%%%%%%%%%%%%%%%%%%%%%%%%%%%%%%%%%%%%%%%%%%%%%%%%%%%%%%%%%%%%%%%%%%%%%%%%%%%%
% Limitations 
%%%%%%%%%%%%%%%%%%%%%%%%%%%%%%%%%%%%%%%%%%%%%%%%%%%%%%%%%%%%%%%%%%%%%%%%%%%%%%%%%%%%%%%%%%
\section{Limitations} \label{sec:limitations}
As with most data-driven techniques, it is hard to definitely declare that the proposed 
method will increase performance in all cases. It is possible that having an extremely poor
analytical model may hurt rather than help the training process. However, we found that even
when the $\alpha$ parameter is extremely small (placing little weight on the Jacobians 
during the learning process), it still dramatically improves the sample efficiency. It is 
also quite possible that the performance gaps between eDMD and jDMD shown here can be 
reduced through better selection of basis functions and better training data sets; however,
given that the proposed approach converges to eDMD as $\alpha \rightarrow 0$, we see no 
reason to not adopt the proposed methodology as simply tune $\alpha$ based on the 
confidence of the model and the quantity (and quality) of training data.


%%%%%%%%%%%%%%%%%%%%%%%%%%%%%%%%%%%%%%%%%%%%%%%%%%%%%%%%%%%%%%%%%%%%%%%%%%%%%%%%%%%%%%%%%%
% Conclusion 
%%%%%%%%%%%%%%%%%%%%%%%%%%%%%%%%%%%%%%%%%%%%%%%%%%%%%%%%%%%%%%%%%%%%%%%%%%%%%%%%%%%%%%%%%%
\section{Conclusion and Future Work} \label{sec:conclusion}
We have presented a simple but powerful extension to eDMD, a model-based method for learning
a bilinear representation of arbitrary dynamical systems, that incoporates derivative 
information from an analytical mode. When combined with a simple linear
MPC policy that projects the learned dynamics back into the original state space, we have 
shown that the resulting pipline can dramatically increase sample efficiency, often 
improving over a nominal MPC policy with just a few sample trajectories. Substantial areas 
for future work remain: most notably testing the proposed pipeline on hardware. Additional 
directions include lifelong learning or adaptive control applications, residual dynamics 
learning, as well as the development of specialized numerical methods for solving nonlinear 
optimal control problems using the learned bilinear dynamics.

\bibliography{references.bib}

\end{document}